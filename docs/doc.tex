\documentclass[10pt,a4paper]{article}

\usepackage[left=2cm,right=2cm,top=2cm,bottom=2cm]{geometry}
\usepackage[utf8]{inputenc}
\usepackage[T1]{fontenc}
\usepackage[unicode]{hyperref}
\usepackage{xcolor}
\newcommand\todo[1]{\textcolor{red}{\textbf{[[ #1 ]]}}}
\usepackage{courier}

\begin{document}
\noindent\textbf{Documentation of Project Implementation for IPP 2019/2020\\
Name and surname: Pavel Yadlouski\\
Login: xyadlo00\\}

\section{Interpret}
\subsection{Script composition}
Interpret comprises from managing script \texttt{interpret.py} and package that
provides all main functionality of interpret. In this package there are 4 other
modules:
\begin{enumerate}
    \item \texttt{opcode.py}
    \item \texttt{frames.py}
    \item \texttt{errors.py}
    \item \texttt{other\_functions.py}
\end{enumerate}

\subsection{Managing script}
Managing script contains two functions (\texttt{main} and \texttt{process\_xml}),
which are called in \textit{if} statement in following order: \texttt{main} 
$\rightarrow$ \texttt{proces\_xml}. In both functions exceptions can be raised 
(return codes and resolved through exceptions), 
so calls of this functions are in \textit{try} block followed by a series of 
\textit{except} blocks in which all possible errors are processed.

\texttt{main} function engaged in processing of input parameters. It checks if 
\texttt{-{}-source} and/or \texttt{-{}-input} are set and, if set, then if the
their values are correct. In case the parameter \texttt{-{}-help} is present,
then check if any other parameter is not used and writes description of script.
This function can raise two exceptions.

\subsection{Modules in package}

\subsubsection{Operation codes}

\subsubsection{secondary functions}

\section{Testing}



\end{document}